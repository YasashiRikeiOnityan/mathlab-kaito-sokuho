%%%%%%%%%%%%%%%%%%%%%%%%%%%%%%%%%%%%%%%%%%%%%%%%%%
% Metadata
%%%%%%%%%%%%%%%%%%%%%%%%%%%%%%%%%%%%%%%%%%%%%%%%%%
% id: 2026-tokyo-1
% title: 2026年 東京大学 理科 第1問
% tags: []
% difficulty: C
% source: https://www.u-tokyo.ac.jp/content/400239118.pdf

%%%%%%%%%%%%%%%%%%%%%%%%%%%%%%%%%%%%%%%%%%%%%%%%%%
% Preamble
%%%%%%%%%%%%%%%%%%%%%%%%%%%%%%%%%%%%%%%%%%%%%%%%%%
\documentclass[fleqn]{ltjsarticle}

\usepackage{common}
\loadcommonpreamble

% ヘッダー
\lhead{\textbf{2026年 東京大学 理科}}

%%%%%%%%%%%%%%%%%%%%%%%%%%%%%%%%%%%%%%%%%%%%%%%%%%
% Document
%%%%%%%%%%%%%%%%%%%%%%%%%%%%%%%%%%%%%%%%%%%%%%%%%%
\begin{document}

\begin{problembox}
    \begin{center}
        {\textbf{第}\hspace{2\zw}\textbf{\textsf{1}}\hspace{2\zw}\textbf{問}}
    \end{center}
    \begin{enumerate}
        \item [\kakkoichi] 函数$f(\theta)=\sin\theta-\theta+\bunsuu{\theta^3}6$の区間$-1\leqq\theta\leqq1$における最大値$M$および最小値$m$を求めよ.
        \item [\kakkoni] (1)で定めた$M$に対し,次の不等式を示せ.
        $$\bunsuu78\pi\leqq\int_0^{2\pi}\sin(\cos x-x)dx\leqq\bunsuu78\pi+4M$$
    \end{enumerate}
\end{problembox}

\begin{multicols}{2}

\noindent
\kakkoichi\quad
$f(\theta)$を微分すると
\begin{align*}
f'(\theta)&=\cos\theta-1+\bunsuu{\theta^2}2\\
f''(\theta)&=-\sin\theta+\theta\\
f'''(\theta)&=-\cos\theta+1
\end{align*}
となる.ここで,常に$f'''(\theta)\geqq0$が成り立つので$f''(\theta)$は単調増加函数である.$f''(0)=0$であるから,$\theta\leqq0$で$f''(\theta)\leqq0$が,$0\leqq\theta$で$f''(\theta)\geqq0$がそれぞれ成り立つ.したがって$f'(\theta)$は$\theta=0$で極小値$f'(0)=0$をとる.ゆえに,常に$0\leqq f'(\theta)$なので,$f(\theta)$は単調増加する.以上より,$-1\leqq\theta\leqq1$における$f(\theta)$の最大値は
\begin{align*}
M=f(1)=\boldsymbol{\sin1-\bunsuu56}
\end{align*}
であり,最小値は
\begin{align*}
m=f(-1)=\boldsymbol{-\sin1+\bunsuu56}
\end{align*}
である.

\noindent
\kakkoni\quad
まず,加法定理から
{\setlength\mathindent{0pt}
    \begin{align*}
    &\phantom{{}={}}\int_0^{2\pi}\sin(\cos x-x)dx\\
    &=\int_0^{2\pi}\sin(\cos x)\cos xdx-\int_0^{2\pi}\cos(\cos x)\sin xdx
    \end{align*}
}%
となる.ここで,右辺第$2$項について$t=x-\pi$と置換すると
{\setlength\mathindent{0pt}
    \begin{align*}
    &\phantom{{}={}}\int_0^{2\pi}\cos(\cos x)\sin xdx\\
    &=\int_{-\pi}^\pi\cos(-\cos t)\cdot(-\sin t)dt\\
    &=-\int_{-\pi}^\pi\cos(\cos t)\sin tdt
    \end{align*}
}%
右辺の被積分函数は奇函数なので,積分区間の対称性から積分値は$0$であることが分かる.よって
{\setlength\mathindent{0pt}
    \begin{align*}
    &\phantom{{}={}}\int_0^{2\pi}\sin(\cos x-x)dx\\
    &=\int_0^{2\pi}\sin(\cos x)\cos xdx\\
    &=\int_0^\pi\sin(\cos x)\cos xdx+\int_\pi^{2\pi}\sin(\cos x)\cos xdx\\
    &=\int_0^\pi\sin(\cos x)\cos xdx\\
    &\qquad{}+\int_0^\pi\sin(-\cos t)\cdot(-\cos t)dt\quad(x=t-\pi)\\
    &=2\int_0^\pi\sin(\cos x)\cos xdx\\
    &=2\int_0^{\pi/2}\sin(\cos x)\cos xdx\\
    &\qquad{}+2\int_0^{\pi/2}\sin(-\cos t)\cdot(-\cos t)dt\quad(x=\pi-t)\\
    &=4\int_0^{\pi/2}\sin(\cos x)\cos xdx
    \end{align*}
}%
の値を評価すればよい.$0\leqq x\leqq\bunsuu\pi2$で$0\leqq\cos x\leqq1$なので,(1)の議論から$0\leqq f(\cos x)\leqq M$であることが分かる.よって
{\setlength\mathindent{0pt}
    \begin{align*}
    0\leqq\int_0^{\pi/2}f(\cos x)\cos xdx\leqq\int_0^{\pi/2}M\cos xdx=M
    \end{align*}
}%
である.$J=\displaystyle\int_0^{\pi/2}\sin(\cos x)\cos xdx$とおくと
{\setlength\mathindent{0pt}
    \begin{align*}
    &\phantom{{}={}}\int_0^{\pi/2}f(\cos x)\cos xdx\\
    &=\int_0^{\pi/2}\left(\sin(\cos x)-\cos x+\bunsuu{\cos^3x}6\right)\cos xdx\\
    &=J-\int_0^{\pi/2}\cos^2xdx+\bunsuu16\int_0^{\pi/2}\cos^4xdx\\
    &=J-\bunsuu12\int_0^{\pi/2}(1+\cos2x)dx\\
    &\qquad{}+\bunsuu1{24}\int_0^{\pi/2}(1+\cos2x)^2dx\\
    &=J-\bunsuu12\left[x-\bunsuu12\sin2x\right]_0^{\pi/2}\\
    &\qquad{}+\bunsuu1{24}\int_0^{\pi/2}(1+2\cos2x+\cos^22x)dx\\
    &=J-\bunsuu\pi4+\bunsuu1{24}\Bigl[x+\sin2x\Bigr]_0^{\pi/2}\\
    &\qquad{}+\bunsuu1{48}\int_0^{\pi/2}(1+\cos4x)dx\\
    &=J-\bunsuu\pi4+\bunsuu\pi{48}+\bunsuu1{48}\left[x+\bunsuu14\sin4x\right]_0^{\pi/2}\\
    &=J-\bunsuu\pi4+\bunsuu\pi{48}+\bunsuu\pi{96}=J-\bunsuu7{32}\pi
    \end{align*}
}%
となる.以上より,$0\leqq J-\bunsuu7{32}\pi\leqq M$なので
{\setlength\mathindent{0pt}
    \begin{align*}
    \bunsuu78\pi\leqq\int_0^{2\pi}\sin(\cos x-x)dx=4J\leqq\bunsuu78\pi+4M
    \end{align*}
}%
を得る.

\end{multicols}

\end{document}