%%%%%%%%%%%%%%%%%%%%%%%%%%%%%%%%%%%%%%%%%%%%%%%%%%
% Metadata
%%%%%%%%%%%%%%%%%%%%%%%%%%%%%%%%%%%%%%%%%%%%%%%%%%
% id: 2026-tokyo-1
% title: 2026年 東京大学 理科 第1問
% tags: []
% difficulty: C
% source: https://www.u-tokyo.ac.jp/content/400239118.pdf

%%%%%%%%%%%%%%%%%%%%%%%%%%%%%%%%%%%%%%%%%%%%%%%%%%
% Preamble
%%%%%%%%%%%%%%%%%%%%%%%%%%%%%%%%%%%%%%%%%%%%%%%%%%
\documentclass[fleqn]{ltjsarticle}

\usepackage{common}
\loadcommonpreamble

% ヘッダー
\lhead{\textbf{2026年 東京大学 理科}}

%%%%%%%%%%%%%%%%%%%%%%%%%%%%%%%%%%%%%%%%%%%%%%%%%%
% Document
%%%%%%%%%%%%%%%%%%%%%%%%%%%%%%%%%%%%%%%%%%%%%%%%%%
\begin{document}

\begin{problembox}
    \begin{center}
        {\textbf{第}\hspace{2\zw}\textbf{\textsf{1}}\hspace{2\zw}\textbf{問}}
    \end{center}
    \begin{enumerate}
        \item [\kakkoichi] 函数$f(\theta)=\sin{\theta}-\theta+\bunsuu{\theta^{3}}6$の区間$-1\leqq\theta\leqq1$における最大値$M$および最小値$m$を求めよ.
        \item [\kakkoni] \kakkoichi で定めた$M$に対し,次の不等式を示せ.
        \begin{align*}   
            \bunsuu{7}{8}\pi\leqq\int_0^{2\pi}\sin(\cos{x}-x)dx\leqq\bunsuu{7}{8}\pi+4M
        \end{align*}
    \end{enumerate}
\end{problembox}

% \begin{multicols}{2}

\noindent
\kakkoichi\quad
$f(\theta)$を微分すると
\begin{align*}
    f^{\prime}(\theta)=\cos{\theta}-1+\bunsuu{\theta^{2}}2,\quad f^{\prime\prime}(\theta)=-\sin{\theta}+\theta,\quad f^{\prime\prime\prime}(\theta)=-\cos{\theta}+1
\end{align*}
となる.ここで,常に$f^{\prime\prime\prime}(\theta)\geqq0$が成り立つので$f^{\prime\prime}(\theta)$は単調増加函数である.$f''(0)=0$であるから,$\theta\leqq0$で$f^{\prime\prime}(\theta)\leqq0$が,$0\leqq\theta$で$f^{\prime\prime}(\theta)\geqq0$がそれぞれ成り立つ.したがって$f^{\prime}(\theta)$は$\theta=0$で極小値$f'(0)=0$をとる.ゆえに,常に$0\leqq f^{\prime}(\theta)$なので,$f(\theta)$は単調増加する.
以上より,
\begin{align*}
    \boldsymbol{M}=f(1)=\boldsymbol{\sin1-\bunsuu{5}{6}},\quad \boldsymbol{m}=f(-1)=\boldsymbol{-\sin1+\bunsuu{5}{6}}
\end{align*}

\noindent
\kakkoni\quad
まず,加法定理から
% {\setlength\mathindent{0pt}
%     \begin{align*}
%     &\phantom{{}={}}\int_0^{2\pi}\sin(\cos{x}-x)dx\\
%     &=\int_0^{2\pi}\sin{(\cos{x})}\sdot\cos{x}\,dx-\int_0^{2\pi}\cos(\cos{x})\sin xdx
%     \end{align*}
% }%
\begin{align*}
    \dint{0}{2\pi}{\sin(\cos{x}-x)\,dx}=\dint{0}{2\pi}{\sin{(\cos{x})}\sdot\cos{x}\,dx}-\dint{0}{2\pi}{\cos{(\cos{x})}\sdot\sin{x}\,dx}
\end{align*}
となる.ここで,右辺第$2$項について$t=x-\pi$と置換すると
% {\setlength\mathindent{0pt}
%     \begin{align*}
%     &\phantom{{}={}}\int_0^{2\pi}\cos(\cos{x})\sin xdx\\
%     &=\int_{-\pi}^\pi\cos(-\cos t)\cdot(-\sin t)dt\\
%     &=-\int_{-\pi}^\pi\cos(\cos t)\sin tdt
%     \end{align*}
% }%
\begin{align*}
    \dint{0}{2\pi}{\cos{(\cos{x})}\sdot\sin{x}\,dx}&=\dint{-\pi}{\pi}{\cos{(-\cos{t})}\sdot(-\sin{t})\,dt}=-\dint{-\pi}{\pi}{\cos{(\cos{t})}\sdot\sin{t}\,dt}
\end{align*}
$\cos{(\cos{t})}\sdot\sin{t}$は奇函数なので,積分区間の対称性から積分値は$0$であることが分かる.よって
% {\setlength\mathindent{0pt}
%     \begin{align*}
%     &\phantom{{}={}}\int_0^{2\pi}\sin(\cos{x}-x)dx\\
%     &=\int_0^{2\pi}\sin{(\cos{x})}\sdot\cos{x}\,dx\\
%     &=\dint{0}{\pi}\sin{(\cos{x})}\sdot\cos{x}\,dx+\int_\pi^{2\pi}\sin{(\cos{x})}\sdot\cos{x}\,dx\\
%     &=\dint{0}{\pi}\sin{(\cos{x})}\sdot\cos{x}\,dx\\
%     &\qquad{}+\dint{0}{\pi}\sin(-\cos t)\cdot(-\cos t)dt\quad(x=t-\pi)\\
%     &=2\dint{0}{\pi}\sin{(\cos{x})}\sdot\cos{x}\,dx\\
%     &=2\dint{0}{\pi/2}\sin{(\cos{x})}\sdot\cos{x}\,dx\\
%     &\qquad{}+2\dint{0}{\pi/2}\sin(-\cos t)\cdot(-\cos t)dt\quad(x=\pi-t)\\
%     &=4\dint{0}{\pi/2}\sin{(\cos{x})}\sdot\cos{x}\,dx
%     \end{align*}
% }%
\begin{align*}
    \dint{0}{2\pi}{\sin{(\cos{x}-x)}\,dx}&=\dint{0}{2\pi}{\sin{(\cos{x})}\sdot\cos{x}\,dx} \\
    &=\dint{0}{\pi}{\sin{(\cos{x})}\sdot\cos{x}\,dx}+\dint{\pi}{2\pi}{\sin{(\cos{x})}\sdot\cos{x}\,dx} \\
    &=\dint{0}{\pi}{\sin{(\cos{x})}\sdot\cos{x}\,dx}+\dint{0}{\pi}{\sin{(-\cos{t})}\sdot(-\cos{t})\,dt}\quad(x=t-\pi) \\
    &=2\dint{0}{\pi}{\sin{(\cos{x})}\sdot\cos{x}\,dx} \\
    &=2\dint{0}{\pi/2}{\sin{(\cos{x})}\sdot\cos{x}\,dx}+2\dint{0}{\pi/2}{\sin{(-\cos{t})}\sdot(-\cos{t})\,dt}\quad(x=\pi-t) \\
    &=4\dint{0}{\pi/2}{\sin{(\cos{x})}\sdot\cos{x}\,dx}
\end{align*}
の値を評価すればよい.$0\leqq x\leqq\bunsuu{\pi}{2}$で$0\leqq\cos{x}\leqq1$なので,\tokeiichi の議論から$0\leqq f(\cos{x})\leqq M$であることが分かる.よって
% {\setlength\mathindent{0pt}
%     \begin{align*}
%     0\leqq\dint{0}{\pi/2}f(\cos{x})\cos{x}\,dx\leqq\dint{0}{\pi/2}M\cos{x}\,dx=M
%     \end{align*}
% }%
\begin{align*}
    0\leqq\dint{0}{\pi/2}f(\cos{x})\cos{x}\,dx\leqq\dint{0}{\pi/2}M\cos{x}\,dx=M
\end{align*}
である.
% {\setlength\mathindent{0pt}
%     \begin{align*}
%     &\phantom{{}={}}\dint{0}{\pi/2}f(\cos{x})\cos{x}\,dx\\
%     &=\dint{0}{\pi/2}\left(\sin{(\cos{x})}-\cos{x}+\bunsuu{\cos^3x}6\right)\cos{x}\,dx\\
%     &=J-\dint{0}{\pi/2}\cos^2xdx+\bunsuu{1}{6}\dint{0}{\pi/2}\cos^4xdx\\
%     &=J-\bunsuu{1}{2}\dint{0}{\pi/2}(1+\cos2x)dx\\
%     &\qquad{}+\bunsuu{1}{24}\dint{0}{\pi/2}(1+\cos2x)^2dx\\
%     &=J-\bunsuu{1}{2}\left[x-\bunsuu{1}{2}\sin2x\right]_0^{\pi/2}\\
%     &\qquad{}+\bunsuu{1}{24}\dint{0}{\pi/2}(1+2\cos2x+\cos^22x)dx\\
%     &=J-\bunsuu{\pi}{4}+\bunsuu{1}{24}\Bigl[x+\sin2x\Bigr]_0^{\pi/2}\\
%     &\qquad{}+\bunsuu{1}{48}\dint{0}{\pi/2}(1+\cos4x)dx\\
%     &=J-\bunsuu{\pi}{4}+\bunsuu{\pi}{48}+\bunsuu{1}{48}\left[x+\bunsuu{1}4\sin4x\right]_0^{\pi/2}\\
%     &=J-\bunsuu{\pi}{4}+\bunsuu{\pi}{48}+\bunsuu\pi{96}=J-\bunsuu7{32}\pi
%     \end{align*}
% }%
\begin{align*}
    \dint{0}{\pi/2}f(\cos{x})\cos{x}\,dx&=\dint{0}{\pi/2}{\left(\sin{(\cos{x})}-\cos{x}+\bunsuu{\cos^3x}{6}\right)\cos{x}\,dx} \\
    &=J-\dint{0}{\pi/2}{\cos^{2}{x}\,dx}+\bunsuu{1}{6}\dint{0}{\pi/2}{\cos^{4}{x}\,dx}\quad \left(J=\dint{0}{\pi/2}\sin{(\cos{x})}\sdot\cos{x}\,dx\right) \\
    &=J-\bunsuu{1}{2}\dint{0}{\pi/2}{(1+\cos2x)\,dx}+\bunsuu{1}{24}\dint{0}{\pi/2}{(1+\cos{2x})^{2}\,dx} \\
    &=J-\bunsuu{1}{2}\teisekibun{x-\bunsuu{1}{2}\sin2x}{0}{\pi/2}+\bunsuu{1}{24}\dint{0}{\pi/2}{(1+2\cos{2x}+\cos^{2}{2x})\,dx} \\
    &=J-\bunsuu{\pi}{4}+\bunsuu{1}{24}\teisekibun{x+\sin2x}{0}{\pi/2}+\bunsuu{1}{48}\dint{0}{\pi/2}{(1+\cos{4x})\,dx} \\
    &=J-\bunsuu{\pi}{4}+\bunsuu{\pi}{48}+\bunsuu{1}{48}\teisekibun{x+\bunsuu{1}4\sin4x}{0}{\pi/2} \\
    &=J-\bunsuu{\pi}{4}+\bunsuu{\pi}{48}+\bunsuu\pi{96} \\
    &=J-\bunsuu7{32}\pi
\end{align*}
となる.以上より,$0\leqq J-\bunsuu{7}{32}\pi\leqq M$なので
% {\setlength\mathindent{0pt}
%     \begin{align*}
%     \bunsuu{7}{8}\pi\leqq\int_0^{2\pi}\sin(\cos{x}-x)dx=4J\leqq\bunsuu{7}{8}\pi+4M
%     \end{align*}
% }%
\begin{align*}
    \boldsymbol{\bunsuu{7}{8}\pi\leqq\dint{0}{2\pi}{\sin{(\cos{x}-x)}\,dx}=4J\leqq\bunsuu{7}{8}\pi+4M}
\end{align*}
を得る.\owari

% \end{multicols}

\end{document}