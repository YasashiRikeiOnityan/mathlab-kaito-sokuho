%%%%%%%%%%%%%%%%%%%%%%%%%%%%%%%%%%%%%%%%%%%%%%%%%%
% Metadata
%%%%%%%%%%%%%%%%%%%%%%%%%%%%%%%%%%%%%%%%%%%%%%%%%%
% id: 2026-kyoto-3
% title: 2026年 京都大学 理系 第3問
% tags: []
% difficulty: C
% source: https://www.kyoto-u.ac.jp/sites/default/files/inline-files/admissionsundergradpast_eqR06_eqdocumentsR06_3M07-b3308c56005228b7b6d1c7e90375f55a.pdf

%%%%%%%%%%%%%%%%%%%%%%%%%%%%%%%%%%%%%%%%%%%%%%%%%%
% Preamble
%%%%%%%%%%%%%%%%%%%%%%%%%%%%%%%%%%%%%%%%%%%%%%%%%%
\documentclass[fleqn]{ltjsarticle}

\usepackage{common}
\loadcommonpreamble

% ヘッダー
\lhead{\textbf{2026年 京都大学 理系}}

%%%%%%%%%%%%%%%%%%%%%%%%%%%%%%%%%%%%%%%%%%%%%%%%%%
% Document
%%%%%%%%%%%%%%%%%%%%%%%%%%%%%%%%%%%%%%%%%%%%%%%%%%
\begin{document}

\begin{problembox}
    \begin{enumerate} 
        \item [\huge \shikakusan]\hfill\raisebox{1ex}{(35点)} \\
        $n$は正の整数とする.整数係数の多項式
        \begin{align*}
        (x+1)^{2^{n+1}}-(x^2+1)^{2^n}
        \end{align*}
        のすべての係数が$2^m$で割り切れるような正の整数$m$のうち,最大のものは$n+1$であることを示せ.
    \end{enumerate}
\end{problembox}

\begin{multicols}{2}

\noindent
$(x+1)^{2^{n+1}}=\Bigl((x+1)^2\Bigr)^{2^n}=\Bigl((x^2+1)+2x\Bigr)^{2^n}$と書きなおせるので,二項定理から
\begin{align*}
&\phantom{{}={}}\Bigl((x^2+1)+2x\Bigr)^{2^n}-(x^2+1)^{2^n}\\
&=\sum_{k=1}^{2^n}{}_{2^n}\mathrm C_k\cdot(x^2+1)^{2^n-k}\cdot(2x)^k
\end{align*}
と書ける.\\
ここで,$1\leqq k\leqq2^n$の範囲にあるすべての整数$k$について${}_{2^n}\mathrm C_k\cdot2^k$が$2^{n+1}$で割り切れることを示す.まず,
\begin{align*}
{}_{2^n}\mathrm C_k\cdot2^k&=\bunsuu{2^n!}{(2^n-k)!\cdot k!}\cdot2^k\\
&=\bunsuu{(2^n-1)!}{(2^n-k)!\cdot(k-1)!}\cdot\bunsuu{2^n}k\cdot2^k\\
&={}_{2^n-1}\mathrm C_{k-1}\cdot\bunsuu{2^k}k\cdot2^n
\end{align*}
である.ここで
\begin{align*}
2^k=\sum_{j=0}^k{}_k\mathrm C_j\geqq\sum_{j=0}^k1=k+1>k>0
\end{align*}
即ち$0<k<2^k$なので,$k$は$2^k$の倍数であり得ない.したがって,$k$に含まれる素因数$2$の個数は$k$より少ない.${}_{2^n-1}\mathrm C_{k-1}$は整数であるので,${}_{2^n}\mathrm C_k\cdot2^k$に含まれる素因数$2$の個数は$k-k+n=n$より多い.つまり,${}_{2^n}\mathrm C_k\cdot2^k$は$2^{n+1}$の倍数である.\\
以上より
\begin{align*}
&\phantom{{}={}}(x+1)^{2^{n+1}}-(x^2+1)^{2^n}\\
&=\sum_{k=1}^{2^n}{}_{2^n}\mathrm C_k\cdot2^k\cdot(x^2+1)^{2^n-k}\cdot x^k
\end{align*}
のすべての係数は$2^{n+1}$で割り切れることが分かる.一方で,$(x+1)^{2^{n+1}}-(x^2+1)^{2^n}$を展開したときの$x$の係数は${}_{2^{n+1}}\mathrm C_1-0=2^{n+1}$なので,条件を満たすような正の整数$m$のうち最大のものは$n+1$であることが分かる.

\end{multicols}

\end{document}