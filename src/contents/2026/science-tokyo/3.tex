%%%%%%%%%%%%%%%%%%%%%%%%%%%%%%%%%%%%%%%%%%%%%%%%%%
% Metadata
%%%%%%%%%%%%%%%%%%%%%%%%%%%%%%%%%%%%%%%%%%%%%%%%%%
% id: 2026-science-tokyo-3
% title: 2026年 東京科学大学 第3問
% tags: []
% difficulty: C
% source: https://admissions.isct.ac.jp/ja/013/undergraduate/examination-questions

%%%%%%%%%%%%%%%%%%%%%%%%%%%%%%%%%%%%%%%%%%%%%%%%%%
% Preamble
%%%%%%%%%%%%%%%%%%%%%%%%%%%%%%%%%%%%%%%%%%%%%%%%%%
\documentclass[fleqn]{ltjsarticle}

\usepackage{common}
\loadcommonpreamble

% ヘッダー
\lhead{\textbf{2026年 東京科学大学}}

%%%%%%%%%%%%%%%%%%%%%%%%%%%%%%%%%%%%%%%%%%%%%%%%%%
% Document
%%%%%%%%%%%%%%%%%%%%%%%%%%%%%%%%%%%%%%%%%%%%%%%%%%
\begin{document}

\begin{problembox}
    \begin{enumerate} 
        \item [\huge \shikakusan]\quad\raisebox{1ex}{(60点)} \\
        \hspace{1em}$a$ と $b$ を正の実数とし,座標平面の3点 $\mathrm{O}(0, 0), \mathrm{A}(a, 0), \mathrm{B}(0, b)$ を考える.
        ABの中点をC, ACの中点をPとし, $\sankaku{BCO}$ の外心をQとする.
        \begin{enumerate}
            \item [\kakkoichi] 2点P, Qのそれぞれの座標を $a$ と $b$ を用いて表せ.
            \item [\kakkoni] 2点O, Aを通る直線上の点Rで,$\kaku{PRQ}=\bunsuu{\pi}{2}$ をみたすもの全てについて,それぞれの座標を $a$ と $b$ を用いて表せ.
            \item [\kakkosan] \kakkoni の条件をみたす点Rで,
                \begin{align*}
                    \sankaku{PQR}\soji\sankaku{ABO},\quad\sankaku{QPR}\soji\sankaku{ABO}
                \end{align*}
                のうち少なくとも一方が成り立つようなものをすべて求めよ.
        \end{enumerate}
    \end{enumerate}
\end{problembox}

\begin{multicols*}{2}

\noindent
\kakkoichi\quad 点Cの座標は $\left(\bunsuu{a}{2}, \bunsuu{b}{2}\right)$ であるから, \\
点Pの座標は
\begin{align*}
    \mathrm{P}=\left(\bunsuu{a+a/2}{2}, \bunsuu{0+b/2}{2}\right)=\boldsymbol{\left(\bunsuu{3a}{4}, \bunsuu{b}{4}\right)}
\end{align*}
また,線分OBの垂直二等分線と線分OCの垂直二等分線の交点が点Qの座標である. それぞれの直線の方程式は
\begin{align*}
    y=\bunsuu{b}{2},~ y=-\bunsuu{a}{b}\left(x-\bunsuu{a}{4}\right)+\bunsuu{b}{4}
\end{align*}
より,点Qの座標は
\begin{align*}
    \mathrm{Q}=\boldsymbol{\left(\bunsuu{a^{2}-b^{2}}{4a}, \bunsuu{b}{2}\right)}
\end{align*}

\begin{center}
\begin{tikzpicture}
    \coordinate (O) at (0,0);
    \coordinate (A) at (3,0);
    \coordinate (B) at (0,2);
    \coordinate (C) at (1.5,1);
    \coordinate (P) at (2.25,0.5);
    \coordinate (Q) at (5/12,1);
    \draw [semithick,->,>=stealth] (-0.75,0) -- (3.5,0) node [below] {$x$};
    \draw [semithick,->,>=stealth] (0,-0.75) -- (0,2.5) node [left] {$y$};
    \fill (O) circle [radius=1.5pt] node [below left] {O};
    \fill (A) circle [radius=2pt] node [below] {A};
    \fill (B) circle [radius=2pt] node [left] {B};
    \fill (C) circle [radius=2pt] node [above right] {C};
    \fill (P) circle [radius=2pt] node [above right] {P};
    \fill (Q) circle [radius=2pt] node [above] {Q};
    \draw (A) -- (B);
    \draw (O) -- (C);
    \draw [dashed] ($(O)!0.5!(B)$) -- (Q) -- ($(O)!0.5!(C)$);
    \coordinate (S) at ($(O)!0.5!(B)$);
    \coordinate (T) at ($(O)!0.5!(C)$);
    \draw ($(S)!6pt!(O)$)--($(S)!6pt!(O)!6pt!90:(O)$)--($(S)!6pt!(Q)$);
    \draw ($(T)!6pt!(C)$)--($(T)!6pt!(C)!6pt!90:(C)$)--($(T)!6pt!(Q)$);
\end{tikzpicture}
\end{center}

\noindent
\kakkoni\quad 図のように $\theta_{\mathrm{P}}, \theta_{\mathrm{Q}}$ と置き,$\theta_{\mathrm{Q}}-\theta_{\mathrm{P}}=\bunsuu{\pi}{2}$ となるとき,すなわち,RPとRQが直交するときを考える.

\begin{center}
\begin{tikzpicture}
    \coordinate (O) at (0,0);
    \coordinate (A) at (3,0);
    \coordinate (B) at (0,2);
    \coordinate (C) at (1.5,1);
    \coordinate (P) at (2.25,0.5);
    \coordinate (Q) at (5/12,1);
    \coordinate (R) at (3/4,0);
    \draw [semithick] (-0.75,0) -- (3.5,0);
    \fill (P) circle [radius=2pt] node [above right] {P};
    \fill (Q) circle [radius=2pt] node [above] {Q};
    \fill (R) circle [radius=2pt] node [below] {R};
    \draw (P) -- (R) -- (Q);
    \draw pic["$\theta_{\mathrm{P}}$", draw=black, angle eccentricity=1.5, angle radius=0.8cm] {angle=A--R--P};
    \draw pic["$\theta_{\mathrm{Q}}$", draw=black, angle eccentricity=1.5, angle radius=0.4cm] {angle=A--R--Q};
\end{tikzpicture}
\end{center}

\noindent
$\mathrm{R}=(k, 0)$ と置くと,
\begin{align*}
    &(\text{RPの傾き})=\bunsuu{b}{3a-4k} \\
    &(\text{RQの傾き})=\bunsuu{2ab}{a^{2}-b^{2}-4ak}
\end{align*}
より,RPとRGが直交する条件は傾きの積が $-1$ となることであるから,
\begin{align*}
    &\bunsuu{b}{3a-4k} \sdot \bunsuu{2ab}{a^{2}-b^{2}-4ak} = -1 \\
    &\yueni~(a-4k)(3a^{2}-4ak-b^{2})=0 \\
    &\yueni~k=\bunsuu{a}{4}, \bunsuu{3a^{2}-b^{2}}{4a}
\end{align*}
したがって,求めるRの座標は
\begin{align*}
    \boldsymbol{\left(\bunsuu{a}{4}, 0\right),~ \left(\bunsuu{3a^{2}-b^{2}}{4a}, 0\right)}
\end{align*}

\noindent
\kakkosan\quad \tokeini で求めた座標をそれぞれ
\begin{align*}
    \text{R}_{1}=\left(\bunsuu{a}{4}, 0\right),~ \text{R}_{2}=\left(\bunsuu{3a^{2}-b^{2}}{4a}, 0\right)
\end{align*}
とする.

\begin{center}
\begin{tikzpicture}
    \coordinate (O) at (0,0);
    \coordinate (A) at (3,0);
    \coordinate (B) at (0,2);
    \coordinate (P) at (2.25,0.5);
    \coordinate (Q) at (5/12,1);
    \coordinate (R1) at (3/4,0);
    \coordinate (R2) at (23/12,0);
    \draw [semithick,->,>=stealth] (-0.75,0) -- (3.5,0) node [below] {$x$};
    \draw [semithick,->,>=stealth] (0,-0.75) -- (0,2.5) node [left] {$y$};
    \fill (O) circle [radius=1.5pt] node [below left] {O};
    \fill (A) circle [radius=2pt] node [below] {A};
    \fill (B) circle [radius=2pt] node [left] {B};
    \fill (P) circle [radius=2pt] node [above right] {P};
    \fill (Q) circle [radius=2pt] node [above] {Q};
    \fill (R1) circle [radius=2pt] node [below] {$\text{R}_{1}$};
    \fill (R2) circle [radius=2pt] node [below] {$\text{R}_{2}$};
    \draw (A) -- (B);
    \draw (Q) -- (R1) -- (P) -- cycle;
    \draw (Q) -- (R2) -- (P) -- cycle;
    \draw ($(R1)!6pt!(P)$)--($(R1)!6pt!(P)!6pt!90:(P)$)--($(R1)!6pt!(Q)$);
    \draw ($(R2)!6pt!(P)$)--($(R2)!6pt!(P)!6pt!90:(P)$)--($(R2)!6pt!(Q)$);
    \draw ($(O)!6pt!(A)$)--($(O)!6pt!(A)!6pt!90:(A)$)--($(O)!6pt!(B)$);
\end{tikzpicture}
\end{center}
\noindent
ここで
\begin{align*}
    &\sankaku{PQR}\soji\sankaku{ABO} \\
    &\iff \tan{\kaku{PQR}}=\tan{\kaku{ABO}}~\left(=\bunsuu{a}{b}\right) \\[1.5mm]
    &\sankaku{QPR}\soji\sankaku{ABO} \\
    &\iff \tan{\kaku{PQR}}=\tan{\kaku{BAO}}~\left(=\bunsuu{b}{a}\right)
\end{align*}
である.

\end{multicols*}

\pagebreak

\begin{multicols*}{2}

\noindent
$\tan{\kaku{PQR}}$ を求めるのに必要な傾きを計算する.
\begin{align*}
    &(\text{QPの傾き})=m_{\text{QP}}=-\bunsuu{ab}{2a^{2}+b^{2}} \\
    &(\text{QR$_1$の傾き})=m_{\text{QR$_1$}}=-\bunsuu{a}{b} \\
    &(\text{QR$_2$の傾き})=m_{\text{QR$_2$}}=-\bunsuu{b}{2a}
\end{align*}
ゆえに,
\begin{align*}
    &\tan{\kaku{PQR$_1$}}=\bunsuu{m_{\text{QP}}-m_{\text{QR$_1$}}}{1+m_{\text{QP}}\sdot m_{\text{QR$_1$}}}=\bunsuu{a}{b} \\[2mm]
    &\tan{\kaku{PQR$_2$}}=\bunsuu{m_{\text{QP}}-m_{\text{QR$_2$}}}{1+m_{\text{QP}}\sdot m_{\text{QR$_2$}}}=\bunsuu{b}{2a}
\end{align*}
\vfill
\null
\columnbreak
\noindent
$\tan{\kaku{PQR$_1$}}$ は常に $\tan{\kaku{ABO}}$ と等しいのでR$_1$ は条件をみたす.R$_2$ は
\begin{align*}
    &\bunsuu{b}{2a}=\bunsuu{a}{b}~\text{または}~\bunsuu{b}{2a}=\bunsuu{b}{a} \iff 2a^{2}=b^{2}
\end{align*}
のとき条件をみたす.$2a^{2}=b^{2}$ のとき,R$_2$ はR$_1$ と一致するので,求める点は $\boldsymbol{\left(\bunsuu{a}{4}, 0\right)}$

\end{multicols*}

\end{document}