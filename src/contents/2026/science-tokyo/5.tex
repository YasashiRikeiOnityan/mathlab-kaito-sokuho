%%%%%%%%%%%%%%%%%%%%%%%%%%%%%%%%%%%%%%%%%%%%%%%%%%
% Metadata
%%%%%%%%%%%%%%%%%%%%%%%%%%%%%%%%%%%%%%%%%%%%%%%%%%
% id: 2026-science-tokyo-5
% title: 2026年 東京科学大学 第5問
% tags: []
% difficulty: C
% source: https://admissions.isct.ac.jp/ja/013/undergraduate/examination-questions

%%%%%%%%%%%%%%%%%%%%%%%%%%%%%%%%%%%%%%%%%%%%%%%%%%
% Preamble
%%%%%%%%%%%%%%%%%%%%%%%%%%%%%%%%%%%%%%%%%%%%%%%%%%
\documentclass[fleqn]{ltjsarticle}

\usepackage{common}
\loadcommonpreamble

% ヘッダー
\lhead{\textbf{2026年 東京科学大学}}

%%%%%%%%%%%%%%%%%%%%%%%%%%%%%%%%%%%%%%%%%%%%%%%%%%
% Document
%%%%%%%%%%%%%%%%%%%%%%%%%%%%%%%%%%%%%%%%%%%%%%%%%%
\begin{document}

\begin{problembox}
    \begin{enumerate} 
        \item [\huge \shikakugo]\quad\raisebox{1ex}{(60点)} \\
        \hspace{1em}$n, k$ を正の整数とする.次の不等式をみたす最小の $k$ を求めよ.
        \begin{align*}
            \dlim{n\to\infty}{\dint{0}{100}{x^{k}e^{-x}\sin^{2}(nx)\,dx}}>10
        \end{align*}
        ただし.$e$ は自然対数の底であり,$e>2$ をみたす.
    \end{enumerate}
\end{problembox}

% \begin{multicols*}{2}

\noindent
半角の公式から
\begin{align*}
    \dint{0}{100}{x^{k}e^{-x}\sin^{2}(nx)\,dx}=\bunsuu{1}{2}\dint{0}{100}{x^{k}e^{-x}\,dx}-\bunsuu{1}{2}\dint{0}{100}{x^{k}e^{-x}\cos(2nx)\,dx}
\end{align*}
第2項を部分積分をすると
\begin{align*}
    &\dint{0}{100}{x^{k}e^{-x}\cos(2nx)\,dx} \\
    &=\teisekibun{x^{k}e^{-x}\sdot\bunsuu{1}{2n}\sin{(2nx)}}{0}{100}-\dint{0}{100}{(kx^{k-1}-x^{k})e^{-x}\sdot\bunsuu{1}{2n}\sin{(2nx)}\,dx} \\
    &=100^{k}e^{-100}\sdot\bunsuu{1}{2n}\sin{(200n)}-\bunsuu{1}{2n}\dint{0}{100}{(kx^{k-1}-x^{k})e^{-x}\sin{(2nx)}\,dx} \\
    &=100^{k}e^{-100}\sdot\bunsuu{1}{2n}\sin{(200n)}-\bunsuu{k}{2n}\dint{0}{100}{x^{k-1}e^{-x}\sin{(2nx)}\,dx}+\bunsuu{1}{2n}\dint{0}{100}{x^{k}e^{-x}\sin{(2nx)}\,dx}
\end{align*}
それぞれの項について,$0\leqq x\leqq 100$ で以下の不等式が成り立つ.
\begin{align*}
    \gauss{1}\quad&-100^{k}e^{-100}\sdot\bunsuu{1}{2n}\leqq 100^{k}e^{-100}\sdot\bunsuu{1}{2n}\sin{(200n)}\leqq 100^{k}e^{-100}\sdot\bunsuu{1}{2n} \\[3mm]
    \gauss{2}\quad&-\bunsuu{k}{2n}\dint{0}{100}{x^{k-1}\,dx}\leqq -\bunsuu{k}{2n}\dint{0}{100}{x^{k-1}e^{-x}\sin{(2nx)}\,dx}\leqq\bunsuu{k}{2n}\dint{0}{100}{x^{k-1}\,dx} \\
    &\yueni~ -\bunsuu{k}{2n}\sdot\bunsuu{100^{k}}{k}\leqq -\bunsuu{k}{2n}\dint{0}{100}{x^{k-1}e^{-x}\sin{(2nx)}\,dx}\leqq\bunsuu{k}{2n}\sdot\bunsuu{100^{k}}{k} \\[3mm]
    \gauss{3}\quad&-\bunsuu{1}{2n}\dint{0}{100}{x^{k}\,dx}\leqq\bunsuu{1}{2n}\dint{0}{100}{x^{k}e^{-x}\sin{(2nx)}\,dx}\leqq\bunsuu{1}{2n}\dint{0}{100}{x^{k}\,dx} \\
    &\yueni~ -\bunsuu{1}{2n}\sdot\bunsuu{100^{k+1}}{k+1}\leqq\bunsuu{1}{2n}\dint{0}{100}{x^{k}e^{-x}\sin{(2nx)}\,dx}\leqq\bunsuu{1}{2n}\sdot\bunsuu{100^{k+1}}{k+1}
\end{align*}
3つの不等式の最左辺,最右辺は全て $n\to\infty$ で0に収束するので,はさみうちの原理から
\begin{align*}
    \dlim{n\to\infty}{\left(\bunsuu{1}{2}\dint{0}{100}{x^{k}e^{-x}\cos(2nx)\,dx}\right)}=0
\end{align*}
よって
\begin{align*}
    \dlim{n\to\infty}{\dint{0}{100}{x^{k}e^{-x}\sin^{2}(nx)\,dx}}=\bunsuu{1}{2}\dint{0}{100}{x^{k}e^{-x}\,dx}
\end{align*}
であるから,$\dint{0}{100}{x^{k}e^{-x}\,dx}>20$ をみたす最小の $k$ を求める.$I_{k}=\dint{0}{100}{x^{k}e^{-x}\,dx}$ とおいて部分積分をすると
\begin{align*}
    I_{k}=\teisekibun{-x^{k}e^{-x}}{0}{100}+\dint{0}{100}{kx^{k-1}e^{-x}\,dx}=kI_{k-1}-100^{k}e^{-100}
\end{align*}
この漸化式と,$I_{0}=\dint{0}{100}{e^{-x}\,dx}=\teisekibun{-e^{-x}}{0}{100}=1-e^{-100}$ から
\begin{alignat*}{2}
    &I_{1}=I_{0}-100e^{-100}=1-101e^{-100}<20,&&\quad I_{2}=2I_{1}-100^{2}e^{-100}=2-10202e^{-100}<20 \\
    &I_{3}=3I_{2}-100^{3}e^{-100}=6-1030606e^{-100}<20,&&\quad I_{4}=4I_{3}-100^{4}e^{-100}=24-104122424e^{-100}>20 \\
    &{}&&\quad\quad ~ (\nazenara ~ 104122424e^{-100}<2^{27}\sdot 2^{-100}=2^{-73}<1)
\end{alignat*}
であるので,条件をみたす最小の $k$ は $\boldsymbol{4}$ である.
% \end{multicols*}

\end{document}