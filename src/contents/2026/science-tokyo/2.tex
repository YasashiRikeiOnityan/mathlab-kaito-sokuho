%%%%%%%%%%%%%%%%%%%%%%%%%%%%%%%%%%%%%%%%%%%%%%%%%%
% Metadata
%%%%%%%%%%%%%%%%%%%%%%%%%%%%%%%%%%%%%%%%%%%%%%%%%%
% id: 2026-science-tokyo-2
% title: 2026年 東京科学大学 第2問
% tags: []
% difficulty: C
% source: https://admissions.isct.ac.jp/ja/013/undergraduate/examination-questions

%%%%%%%%%%%%%%%%%%%%%%%%%%%%%%%%%%%%%%%%%%%%%%%%%%
% Preamble
%%%%%%%%%%%%%%%%%%%%%%%%%%%%%%%%%%%%%%%%%%%%%%%%%%
\documentclass[fleqn]{ltjsarticle}

\usepackage{common}
\loadcommonpreamble

% ヘッダー
\lhead{\textbf{2026年 東京科学大学}}

\newlength{\defaultmathindent}

%%%%%%%%%%%%%%%%%%%%%%%%%%%%%%%%%%%%%%%%%%%%%%%%%%
% Document
%%%%%%%%%%%%%%%%%%%%%%%%%%%%%%%%%%%%%%%%%%%%%%%%%%
\begin{document}

\begin{problembox}
    \begin{enumerate} 
        \item [\huge \shikakuni]\quad\raisebox{1ex}{(60点)} \\
        \hspace{1em}$n$ は4以上の整数であり,$x, y, z$ はすべて正の整数であるとする.
        \begin{enumerate}
            \item [\kakkoichi] $1\leqq r<n$ をみたす整数 $r$ に対して,
                \begin{align*}
                    \kumiawase{n}{r}=\kumiawase{n-1}{r}+\kumiawase{n-1}{r-1}
                \end{align*}
                および
                \begin{align*}
                    \kumiawase{n+1}{r+1}=\sum_{k=r}^{n}\kumiawase{k}{r}
                \end{align*}
                が成り立つことを示せ.
            \item [\kakkoni] 空間の点 $(x, y, z)$ で,
                \begin{align*}
                    x+y+z<n
                \end{align*}
                をみたすものの個数を $n$ を用いて表せ.
            \item [\kakkosan] 空間の点 $(x, y, z)$ で,
                \begin{align*}
                    x+y+z=3n\quad\text{かつ}\quad x<y<z
                \end{align*}
                をみたすものの個数を $n$ を用いて表せ.
        \end{enumerate}
    \end{enumerate}
\end{problembox}

\begin{multicols*}{2}

\noindent
\kakkoichi\quad それぞれ示す.
\begin{align*}
    &\kumiawase{n-1}{r}+\kumiawase{n-1}{r-1} \\
    &=\bunsuu{(n-1)!}{r!(n-r-1)!}+\bunsuu{(n-1)!}{(n-1)!(n-r)!} \\
    &=\bunsuu{(n-1)!(n-r)+(n-1)!r}{r!(n-r)!} \\
    &=\bunsuu{(n-1)!n}{r!(n-r)!} \\
    &=\bunsuu{n!}{r!(n-r)!} \\
    &=\kumiawase{n}{r}\owari
\end{align*}
\begin{align*}
    &\sum_{k=r}^{n}\kumiawase{k}{r} \\
    &=\kumiawase{r}{r}+\kumiawase{r+1}{r}+\kumiawase{r+2}{r}+ \sdots +\kumiawase{n}{r} \\
    &=(\kumiawase{r+1}{r+1}+\kumiawase{r+1}{r})+\kumiawase{r+2}{r}+ \sdots +\kumiawase{n}{r} \\
    &=(\kumiawase{r+2}{r+1}+\kumiawase{r+2}{r})+ \sdots +\kumiawase{n}{r} \\
    &=\kumiawase{r+3}{r+1}+ \sdots +\kumiawase{n}{r} \\
    &=\sdots \\
    &=\kumiawase{n}{r+1}+\kumiawase{n}{r} \\
    &=\kumiawase{n+1}{r+1}\owari
\end{align*}

\noindent
\kakkoni\quad $k=3, 4, \ldots , n-1$ に対し $x+y+z=k$ をみたす $(x, y, z)$ の個数は
\begin{align*}
    (x-1)+(y-1)+(z-1)=k-3
\end{align*}
から $\kumiawase{k-1}{2}$ 個である.よって,求める個数は
\begin{align*}
    \sum_{k=3}^{n-1}\kumiawase{k-1}{2}=\sum_{k=2}^{n-2}\kumiawase{k}{2}=\boldsymbol{\kumiawase{n-1}{3}}~\text{\bf 個}
\end{align*}

\noindent
\kakkosan\quad 全体集合 $U$ を次のように定義する.
{\setlength{\mathindent}{0pt}
\begin{align*}
    U=\{(x, y, z) ~|~ x+y+z=3n,~ x,y,z\in\mathbb{N}\}
\end{align*}}
\noindent
このとき
\begin{align*}
    n(U)=\kumiawase{3n-1}{2}=(3n-1)(3n-2)/2
\end{align*}
である.余事象を考える.

\begin{enumerate}
    \item [\tokeiichi] $x=y=z$ の場合 \\
        $(x, y, z) = (n, n, n)$ の1つがある.
    \item [\tokeini] $x, y, z$ のうち,2つのみが同じ値の場合 \\
        $x=y\neq z$ のときを考える.
        \begin{itemize}
            \item $n$ が奇数のとき \\
                $x$ のとり得る値は $1$ から $(3n-1)/2$ までであり,\tokeiichi の場合を
                除くと,とり得る点の個数は $(3n-1)/2-1=(3n-3)/2$ 個である.
            \item $n$ が偶数のとき \\
                $x$ のとり得る値は $1$ から $(3n-2)/2$ までであり,\tokeiichi の場合を
                除くと,とり得る点の個数は $(3n-2)/2-1=(3n-4)/2$ 個である.
        \end{itemize}
        同様に,$y=z\neq x, z=x\neq y$ の場合も計算できるので,$n$ が奇数のとき $3(3n-3)/2$ 個,$n$ が偶数のとき $3(3n-4)/2$ 個である.
\end{enumerate}

\end{multicols*}

\pagebreak

\begin{multicols*}{2}

\noindent
よって,$x, y, z$ の3つの座標がすべて異なる場合の数は, 
$n$ が奇数のとき,
\begin{align*}
    &(3n-1)(3n-2)/2-3(3n-3)/2-1 \\
    &=(9n^{2}-18n+9)/2 ~\text{個}
\end{align*}
$n$ が偶数のとき,
\begin{align*}
    &(3n-1)(3n-2)/2-3(3n-4)/2-1 \\
    &=(9n^{2}-18n+12)/2 ~\text{個}
\end{align*}
\vfill
\null
\columnbreak
\noindent
$x<y<z$ をみたす場合の数は左の場合からさらに $1/6$ になるので,
$n$ が奇数のとき,
\begin{align*}
    (9n^{2}-18n+9)/12=\boldsymbol{3(n-1)^2/4} ~\text{\bf 個}
\end{align*}
$n$ が偶数のとき,
\begin{align*}
    (9n^{2}-18n+12)/12=\boldsymbol{(3n^{2}-6n+4)/4} ~\text{\bf 個}
\end{align*}

\end{multicols*}

\end{document}