%%%%%%%%%%%%%%%%%%%%%%%%%%%%%%%%%%%%%%%%%%%%%%%%%%
% Metadata
%%%%%%%%%%%%%%%%%%%%%%%%%%%%%%%%%%%%%%%%%%%%%%%%%%
% id: 2026-science-tokyo-4
% title: 2026年 東京科学大学 第4問
% tags: []
% difficulty: C
% source: https://admissions.isct.ac.jp/ja/013/undergraduate/examination-questions

%%%%%%%%%%%%%%%%%%%%%%%%%%%%%%%%%%%%%%%%%%%%%%%%%%
% Preamble
%%%%%%%%%%%%%%%%%%%%%%%%%%%%%%%%%%%%%%%%%%%%%%%%%%
\documentclass[fleqn]{ltjsarticle}

\usepackage{common}
\loadcommonpreamble

% ヘッダー
\lhead{\textbf{2026年 東京科学大学}}

\newcommand{\e}[1]{e\left(#1\pi\right)}

%%%%%%%%%%%%%%%%%%%%%%%%%%%%%%%%%%%%%%%%%%%%%%%%%%
% Document
%%%%%%%%%%%%%%%%%%%%%%%%%%%%%%%%%%%%%%%%%%%%%%%%%%
\begin{document}

\begin{problembox}
    \begin{enumerate} 
        \item [\huge \shikakushi]\quad\raisebox{1ex}{(60点)} \\
        \hspace{1em}$i$ を虚数単位とする.実数 $\theta$ に対して複素数 $e(\theta)$ を
        \begin{align*}
            e(\theta)=\cos{\theta}+i\sin{\theta}
        \end{align*}
        で定める.正の整数 $n$ に対し,複素数平面上の点 $\e{\bunsuu{n-1}{3}}$ と点 $\e{\bunsuu{n}{3}}$ を通る直線を $l_n$ とする.
        複素数 $z$ に対し,直線 $l_n$ に関して点 $z$ と対称な位置にある点を表す複素数を $R_{n}(z)$ とする.
        \begin{enumerate}
            \item [\kakkoichi] 複素数 $z$ と共役な複素数を $\kyouyaku{z}$ とする.このとき,
                \begin{align*}
                    R_{n}(z)=e(\theta_{1})\kyouyaku{z}+\sqrt{3} e(\theta_{2})
                \end{align*}
                をみたす実数 $\theta_{1}, \theta_{2}$ をそれぞれ $n$ を用いて表せ.
            \item [\kakkoni] 実数 $a, b$ に対して,
                \begin{align*}
                    z_{1}=a+bi,\quad z_{n+1}=R_{n}(z_{n})~~(n=1, 2, 3, \ldots)
                \end{align*}
                で定められる複素数の列 $\suuretu{z_{n}}$ の一般項を求めよ.
        \end{enumerate}
    \end{enumerate}
\end{problembox}

% \begin{multicols*}{2}

\noindent
\kakkoichi\quad 適当に点を打って状況を整理すると以下の図の通りである.
平行移動と回転を行って $l_{n}$ が実軸に重なるように変換する.
このとき,$z$ が移った点に対する共役の複素数が $R_{n}(z)$ が移る点である.
この点に対して,行った変換の逆操作を行うことで $R_{n}(z)$ が求まる.

\begin{center}
\begin{minipage}[c]{0.48\linewidth}
\centering
\begin{tikzpicture}[scale=1]
    \coordinate (O) at (0,0);
    \coordinate (A) at ({cos(15)},{sin(15)});
    \coordinate (B) at ({cos(75)},{sin(75)});
    \coordinate (Z) at (1/5,1/5);
    \coordinate (RZ) at ($(Z)!2!($(A)!(Z)!(B)$)$);
    \draw [semithick,->,>=stealth] (-1.2,0) -- (2.2,0) node [below] {Re};
    \draw [semithick,->,>=stealth] (0,-1.2) -- (0,2.2) node [left] {Im};
    \fill (O) circle [radius=1.5pt] node [below left] {O};
    \draw [dotted] (O) circle [radius=1cm];
    \fill (A) circle [radius=1.5pt] ++ (0,0.1) node [right] {\scriptsize $\e{\bunsuu{n-1}{3}}$};
    \fill (B) circle [radius=1.5pt] ++ (-0.2,0.1) node [above right] {\scriptsize $\e{\bunsuu{n}{3}}$};
    \fill (Z) circle [radius=1.5pt] node [right] {$z$};
    \fill (RZ) circle [radius=1.5pt] node [right] {$R_{n}(z)$};
    \draw ($(A)!-1.4!(B)$) -- ($(B)!-1.4!(A)$) node [left] {$l_{n}$};
    \draw [dotted] (Z) -- (RZ);
\end{tikzpicture}
\end{minipage}\hfill
\begin{minipage}[c]{0.48\linewidth}
\centering
\begin{tikzpicture}[scale=1]
    \coordinate (O) at (0,0);
    \coordinate (A) at ({cos(15)},{sin(15)});
    \coordinate (B) at ({cos(75)},{sin(75)});
    \coordinate (Bp) at ($(B)-(A)$);
    \coordinate (Z) at (1/5,1/5);
    \coordinate (RZ) at ($(Z)!2!($(A)!(Z)!(B)$)$);
    \coordinate (Zp) at ($(Z)-(A)$);
    \coordinate (RZp) at ($(RZ)-(A)$);
    \pgfmathsetmacro{\lineangle}{-45}
    % Zpp = Zp を原点中心に -\lineangle 回転した点(回転公式で明示的に計算)
    \path let \p1=($(Zp)-(0,0)$) in \pgfextra{%
      \pgfmathparse{\x1/28.45274}\xdef\Zpx{\pgfmathresult}%
      \pgfmathparse{\y1/28.45274}\xdef\Zpy{\pgfmathresult}%
    } (0,0);
    \pgfmathsetmacro{\Zppx}{\Zpx*cos(-\lineangle) - \Zpy*sin(-\lineangle)}
    \pgfmathsetmacro{\Zppy}{\Zpx*sin(-\lineangle) + \Zpy*cos(-\lineangle)}
    \coordinate (Zpp) at (\Zppx,\Zppy);
    \path let \p1=($(RZp)-(0,0)$) in \pgfextra{%
      \pgfmathparse{\x1/28.45274}\xdef\RZpx{\pgfmathresult}%
      \pgfmathparse{\y1/28.45274}\xdef\RZpy{\pgfmathresult}%
    } (0,0);
    \pgfmathsetmacro{\RZppx}{\RZpx*cos(-\lineangle) - \RZpy*sin(-\lineangle)}
    \pgfmathsetmacro{\RZppy}{\RZpx*sin(-\lineangle) + \RZpy*cos(-\lineangle)}
    \coordinate (RZpp) at (\RZppx,\RZppy);
    \draw [semithick,->,>=stealth] (-1.2,0) -- (2,0) node [below] {Re};
    \draw [semithick,->,>=stealth] (0,-1.2) -- (0,2) node [left] {Im};
    \fill (O) circle [radius=1.5pt];
    \fill (Z) circle [radius=1.5pt] node [above] {$z$};
    \fill (RZ) circle [radius=1.5pt] node [right] {$R_{n}(z)$};
    \fill (A) circle [radius=1.5pt] ++ (0,0.1) node [right] {\scriptsize $\e{\bunsuu{n-1}{3}}$};
    \fill (Zp) circle [radius=1.5pt];
    \fill (RZp) circle [radius=1.5pt];
    \draw ($(A)!-1.4!(B)$) -- ($(B)!-1.4!(A)$) node [left] {$l_{n}$};
    \draw [dashed] ($(O)!-1.7!(Bp)$) -- ($(Bp)!-1.1!(O)$);
    \draw [gray, semithick, ->] (\lineangle:0.7) arc (\lineangle:0:0.7);
    \draw [->, gray] (A) -- ($(A)!0.48!(O)$);
    \draw [->, gray] (A) -- ($(A)!0.52!(O)$);
    \draw [gray] (A) -- (O);
    \draw [->, gray] (Z) -- ($(Z)!0.48!(Zp)$);
    \draw [->, gray] (Z) -- ($(Z)!0.52!(Zp)$);
    \draw [gray] (Z) -- (Zp);
    \draw [->, gray] (RZ) -- ($(RZ)!0.48!(RZp)$);
    \draw [->, gray] (RZ) -- ($(RZ)!0.52!(RZp)$);
    \draw [gray] (RZ) -- (RZp);
    \fill (Zpp) circle [radius=1.5pt];
    \path let \p1=($(Zp)-(0,0)$) in \pgfextra{\pgfmathparse{veclen(\x1,\y1)/28.45274}\xdef\rZprime{\pgfmathresult}\pgfmathparse{atan2(\y1,\x1)}\xdef\startangle{\pgfmathresult}} (0,0);
    \pgfmathsetmacro{\endangle}{\startangle - \lineangle}
    \draw [gray, semithick, ->] (Zp) arc (\startangle:\endangle:\rZprime);
    \fill (RZpp) circle [radius=1.5pt];
    \draw [dotted] (Zpp) -- (RZpp);
    \path let \p1=($(RZp)-(0,0)$) in \pgfextra{\pgfmathparse{veclen(\x1,\y1)/28.45274}\xdef\rRZprime{\pgfmathresult}\pgfmathparse{atan2(\y1,\x1)}\xdef\RZstartangle{\pgfmathresult}} (0,0);
    \pgfmathsetmacro{\RZendangle}{\RZstartangle - \lineangle}
    \draw [gray, semithick, ->] (RZp) arc (\RZstartangle:\RZendangle:\rRZprime);
\end{tikzpicture}
\end{minipage}
\end{center}

\noindent
全体を $-\e{\bunsuu{n-1}{3}}$ 平行移動すると,点 $z$ は $z-\e{\bunsuu{n-1}{3}}$ へ,
点 $\e{\bunsuu{n}{3}}$ は
\begin{align*}
    \e{\bunsuu{n}{3}}-\e{\bunsuu{n-1}{3}}&=\e{\bunsuu{n}{3}}-\e{\bunsuu{n}{3}}\sdot \e{-\bunsuu{1}{3}} \\
    &=\left(1-\bunsuu{1}{2}+\bunsuu{\sqrt{3}}{2}i\right)\sdot \e{\bunsuu{n}{3}} \\
    &=\left(\bunsuu{1}{2}+\bunsuu{\sqrt{3}}{2}i\right)\sdot \e{\bunsuu{n}{3}} \\
    &=\e{\bunsuu{1}{3}}\sdot \e{\bunsuu{n}{3}} \\
    &=\e{\bunsuu{n+1}{3}}
\end{align*}
へ移動する.さらに全体を $-\bunsuu{n+1}{3}\pi$ 回転移動すると, \\
点 $z-\e{\bunsuu{n-1}{3}}$ は
\begin{align*}
    \left(z-\e{\bunsuu{n-1}{3}}\right)\sdot \e{-\bunsuu{n+1}{3}}~\sdots\sdots~\maruichi
\end{align*}
へ移動する.$\maruichi$ に対する共役の複素数は
\begin{align*}
    \kyouyaku{\left(z-\e{\bunsuu{n-1}{3}}\right)\sdot \e{-\bunsuu{n+1}{3}}}&=\left(\kyouyaku{z}-\e{-\bunsuu{n-1}{3}}\right)\sdot \e{\bunsuu{n+1}{3}} \\
    &=\kyouyaku{z}\sdot \e{\bunsuu{n+1}{3}}-\e{\bunsuu{2}{3}}~\sdots\sdots~\maruni
\end{align*}
であり,$\maruni$ は $R_{n}(z)$ が移る点であるから,平行移動,回転移動した分だけ元に戻すと,
% {\setlength{\mathindent}{0pt}
\begin{align*}
    R_{n}(z)&=\left(\kyouyaku{z}\sdot \e{\bunsuu{n+1}{3}}-\e{\bunsuu{2}{3}}\right)\sdot \e{\bunsuu{n+1}{3}}+\e{\bunsuu{n-1}{3}} \\
    &=\kyouyaku{z}\sdot \e{\bunsuu{2(n+1)}{3}}-\e{\bunsuu{n+3}{3}}+\e{\bunsuu{n-1}{3}} \\
    &=\kyouyaku{z}\sdot \e{\bunsuu{2(n+1)}{3}}-\left(\e{\bunsuu{4}{3}}-1\right)\sdot \e{\bunsuu{n-1}{3}} \\
    &=\kyouyaku{z}\sdot \e{\bunsuu{2(n+1)}{3}}+\left(\bunsuu{3}{2}+\bunsuu{\sqrt{3}}{2}i\right)\sdot \e{\bunsuu{n-1}{3}} \\
    &=\kyouyaku{z}\sdot \e{\bunsuu{2(n+1)}{3}}+\sqrt{3}\sdot \e{\bunsuu{1}{6}}\sdot \e{\bunsuu{n-1}{3}} \\
    &=\kyouyaku{z}\sdot \e{\bunsuu{2(n+1)}{3}}+\sqrt{3}\sdot \e{\bunsuu{2n-1}{6}}
\end{align*}%}
を得る.したがって,
\begin{align*}
    \boldsymbol{\theta_{1}=\bunsuu{2(n+1)}{3}\pi},\quad\boldsymbol{\theta_{2}=\bunsuu{2n-1}{6}\pi}
\end{align*}

% \end{multicols*}

\noindent
\kakkoni\quad \kakkoichi の結果から
\begin{align*}
    z_{n+2}&=\e{\bunsuu{2(n+2)}{3}}\sdot \kyouyaku{z_{n+1}}+\sqrt{3}\sdot \e{\bunsuu{2n+1}{6}} \\
    &=\e{\bunsuu{2(n+2)}{3}}\sdot \left(\e{-\bunsuu{2(n+1)}{3}}\sdot z_{n}+\sqrt{3}\sdot \e{-\bunsuu{2n-1}{6}}\right)+\sqrt{3}\sdot\e{\bunsuu{2n+1}{6}} \\
    &=\e{\bunsuu{2}{3}}\sdot z_{n}+\sqrt{3}\sdot\e{\bunsuu{2n+9}{6}}+\sqrt{3}\sdot\e{\bunsuu{2n+1}{6}} \\
    &=\e{\bunsuu{2}{3}}\sdot z_{n}+\sqrt{3}\sdot\e{\bunsuu{4}{3}}\sdot\e{\bunsuu{2n+1}{6}}+\sqrt{3}\sdot\e{\bunsuu{2n+1}{6}} \\
    &=\e{\bunsuu{2}{3}}\sdot z_{n}+\sqrt{3}\left(\e{\bunsuu{4}{3}}+1\right)\sdot\e{\bunsuu{2n+1}{6}} \\
    &=\e{\bunsuu{2}{3}}\sdot z_{n}+\sqrt{3}\left(\bunsuu{1}{2}-\bunsuu{\sqrt{3}}{2}i\right)\sdot\e{\bunsuu{2n+1}{6}} \\
    &=\e{\bunsuu{2}{3}}\sdot z_{n}+\sqrt{3}\sdot\e{\bunsuu{5}{3}}\sdot\e{\bunsuu{2n+1}{6}} \\
    &=\e{\bunsuu{2}{3}}\sdot z_{n}+\sqrt{3}\sdot\e{\bunsuu{2n+11}{6}}
\end{align*}
$n=2m-1~(m\in\mathbb{Z})$ のとき
\begin{alignat*}{2}
    & z_{2m+1}
    &&=\e{\bunsuu{2}{3}}\sdot z_{2m-1}+\sqrt{3}\sdot\e{\bunsuu{4m+9}{6}} \\
    & \bunsuu{z_{2m+1}}{\left(\e{\bunsuu{2}{3}}\right)^{m+1}}
    &&=\bunsuu{z_{2m-1}}{\left(\e{\bunsuu{2}{3}}\right)^{m}}+\sqrt{3}\sdot\e{\bunsuu{5}{6}} \\
    & \bunsuu{z_{2(m+1)-1}}{\e{\bunsuu{2(m+1)}{3}}}
    &&=\bunsuu{z_{2m-1}}{\e{\bunsuu{2m}{3}}}+\sqrt{3}\sdot\e{\bunsuu{5}{6}}
\end{alignat*}
より,数列 $\left\{\bunsuu{z_{2m-1}}{e(2m\pi/3)}\right\}$ は等差数列であるから
\begin{alignat*}{2}
    & \bunsuu{z_{2m-1}}{\e{\bunsuu{2m}{3}}}
    &&=\bunsuu{z_{1}}{\e{\bunsuu{2}{3}}}+\sqrt{3}\sdot\e{\bunsuu{5}{6}}\sdot (m-1) \\
    & z_{2m-1}
    &&=(a+bi)\sdot\e{\bunsuu{2(m-1)}{3}}+\sqrt{3}(m-1)\sdot\e{\bunsuu{4m+5}{6}}
\end{alignat*}
$m=\bunsuu{n+1}{2}$ を代入して
\begin{align*}
    \boldsymbol{z_{n}=(a+bi)\sdot\e{\bunsuu{n-1}{3}}+\bunsuu{\sqrt{3}(n-1)}{2}\sdot\e{\bunsuu{2n+7}{6}}\quad(n\text{\bf は奇数})}
\end{align*}
同様に,$n=2m~(m\in\mathbb{Z})$ のとき
\begin{alignat*}{2}
    & z_{2m+2}
    &&=\e{\bunsuu{2}{3}}\sdot z_{2m}+\sqrt{3}\sdot\e{\bunsuu{4m+11}{6}} \\
    & \bunsuu{z_{2m+2}}{\left(\e{\bunsuu{2}{3}}\right)^{m+1}}
    &&=\bunsuu{z_{2m}}{\left(\e{\bunsuu{2}{3}}\right)^{m}}+\sqrt{3}\sdot\e{\bunsuu{7}{6}} \\
    & \bunsuu{z_{2(m+1)}}{\e{\bunsuu{2(m+1)}{3}}}
    &&=\bunsuu{z_{2m}}{\e{\bunsuu{2m}{3}}}+\sqrt{3}\sdot\e{\bunsuu{7}{6}}
\end{alignat*}
より,数列 $\left\{\bunsuu{z_{2m}}{e(2m\pi/3)}\right\}$ は等差数列であるから
\begin{alignat*}{2}
    & \bunsuu{z_{2m}}{\e{\bunsuu{2m}{3}}}
    &&=\bunsuu{z_{2}}{\e{\bunsuu{2}{3}}}+\sqrt{3}\sdot\e{\bunsuu{7}{6}}\sdot (m-1) \\
    & z_{2m}
    &&=\left((a-bi)\sdot\e{\bunsuu{4}{3}}+\sqrt{3}\sdot\e{\bunsuu{1}{6}}\right)\sdot\e{\bunsuu{2(m-1)}{3}}+\sqrt{3}(m-1)\sdot\e{\bunsuu{4m+7}{6}} \\
    &
    &&=(a-bi)\sdot\e{\bunsuu{2m+2}{3}}+\sqrt{3}\sdot\e{\bunsuu{4m-3}{6}}+\sqrt{3}(m-1)\sdot\e{\bunsuu{4m+7}{6}}
\end{alignat*}
$m=\bunsuu{n}{2}$ を代入して
\begin{align*}
    \boldsymbol{z_{n}=(a-bi)\sdot\e{\bunsuu{n+2}{3}}+\sqrt{3}\sdot\e{\bunsuu{2n-3}{6}}+\bunsuu{\sqrt{3}(n-2)}{2}\sdot\e{\bunsuu{2n+7}{6}}\quad(n\text{\bf は偶数})}
\end{align*}
\end{document}