%%%%%%%%%%%%%%%%%%%%%%%%%%%%%%%%%%%%%%%%%%%%%%%%%%
% Metadata
%%%%%%%%%%%%%%%%%%%%%%%%%%%%%%%%%%%%%%%%%%%%%%%%%%
% id: 2026-science-tokyo-1
% title: 2026年 東京科学大学 第1問
% tags: []
% difficulty: C
% source: https://admissions.isct.ac.jp/ja/013/undergraduate/examination-questions

%%%%%%%%%%%%%%%%%%%%%%%%%%%%%%%%%%%%%%%%%%%%%%%%%%
% Preamble
%%%%%%%%%%%%%%%%%%%%%%%%%%%%%%%%%%%%%%%%%%%%%%%%%%
\documentclass[fleqn]{ltjsarticle}

\usepackage{common}
\loadcommonpreamble

% ヘッダー
\lhead{\textbf{2026年 東京科学大学}}

%%%%%%%%%%%%%%%%%%%%%%%%%%%%%%%%%%%%%%%%%%%%%%%%%%
% Document
%%%%%%%%%%%%%%%%%%%%%%%%%%%%%%%%%%%%%%%%%%%%%%%%%%
\begin{document}

\begin{problembox}
    \begin{enumerate} 
        \item [\huge \shikakuichi]\quad\raisebox{1ex}{(60点)}
        \begin{enumerate}
            \item [\kakkoichi] $x$ は無理数であり,有理数 $a, b$ および正の整数 $n$ により
                $x=a+b\sqrt{n}$ と表されるとする.このとき,$x^{2}+px+q=0$ が成り立つような有理数 $p, q$
                を求めよ.また,そのような有理数 $p, q$ の組はただ一つに限ることを示せ.
            \item [\kakkoni] $x$ は無理数であり,有理数 $a, b$ および正の整数 $n$ により
                $x=a+b\sqrt[3]{n}$ と表されるとする.このとき,$x^{3}+px^{2}+qx+r=0$ が成り立つような有理数 $p, q, r$
                を求めよ,また,そのような有理数 $p, q, r$ の組はただ一つに限ることを示せ.
        \end{enumerate}
    \end{enumerate}
\end{problembox}

\begin{multicols*}{2}

\noindent
\kakkoichi\quad $x-a=b\sqrt{n}$ の両辺を2乗して
\begin{align*}
    &(x-a)^{2}=b^{2}n \\
    &x^{2}-2ax+a^{2}=b^{2}n \\
    &x^{2}-2ax+(a^{2}-b^{2}n)=0
\end{align*}
これを $x^{2}+px+q=0$ と係数比較すると
\begin{align*}
    \boldsymbol{p=-2a,\quad q=a^{2}-b^{2}n}
\end{align*}
これらの $p, q$ は有理数である.他に有理数 $p^{\prime}, q^{\prime}$ が存在して
$x^{2}+p^{\prime}x+q^{\prime}=0$ が成り立つと仮定すると,
\begin{align*}
    (p-p^{\prime})x+(q-q^{\prime})=0~\sdots~\astab
\end{align*}
を得る.$p-p^{\prime}\neq 0$と仮定すると,$x=-\bunsuu{q-q^{\prime}}{p-p^{\prime}}$ 
となり,これは有理数であるから $x$ が無理数であることに矛盾する.
したがって,$p=p^{\prime}$ であり,これを $\astab$ に代入すると $q=q^{\prime}$ を得る.
よって,$p, q$ の一意性が示された.\owari
\vfill
\null
\columnbreak
\noindent
\kakkoni\quad $x-a=b\sqrt[3]{n}$ の両辺を3乗して
\begin{align*}
    &(x-a)^{3}=b^{3}n \\
    &x^{3}-3ax^{2}+3a^{2}x-a^{3}=b^{3}n \\
    &x^{3}-3ax^{2}+3a^{2}x-(a^{3}+b^{3}n)=0
\end{align*}
これを $x^{3}+px^{2}+qx+r=0$ と係数比較すると
\begin{align*}
    \boldsymbol{p=-3a,\quad q=3a^{2},\quad r=-(a^{3}+b^{3}n)}
\end{align*}
これらの $p, q, r$ は有理数である.他に有理数 $p^{\prime}, q^{\prime}, r^{\prime}$ が存在して
$x^{3}+p^{\prime}x^{2}+q^{\prime}x+r^{\prime}=0$ が成り立つと仮定すると,
\begin{align*}
    (p-p^{\prime})x^{2}+(q-q^{\prime})x+(r-r^{\prime})=0~\sdots~\astab
\end{align*}
を得る.$\astab$ において,係数のいずれかが $0$ でないとすると,$x$ は2次以下の有理数係数の方程式を満たすことになるが,
これは成り立たない(証明後述).したがって,
\begin{align*}
    p=p^{\prime}, q=q^{\prime}, r=r^{\prime}
\end{align*}
となり,$p, q, r$ の一意性が示された.\owari

\end{multicols*}

\end{document}